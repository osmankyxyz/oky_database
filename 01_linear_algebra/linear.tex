\documentclass{../00_main/tufte-handout}

%%% STYLES
\usepackage{amsthm}
\usepackage{mdframed}

% Define theorem style
\newtheoremstyle{framed}
  {0pt}{0pt}
  {\normalfont}
  {}
  {\bfseries}
  {.}
  {.5em}
  {}

\theoremstyle{framed}
\newtheorem{theorem}{Theorem}[section]
\newtheorem{lemma}[theorem]{Lemma}
\newtheorem{remark}[theorem]{Remark}
\newtheorem{definition}[theorem]{Definition}
\newtheorem{example}[theorem]{Example}
%\usepackage[style=numeric]{biblatex}
%\addbibresource{nuclear.bib}
% Surround with frame
\surroundwithmdframed[
  linewidth=1pt,
  linecolor=black,
  backgroundcolor=gray!10,
  innertopmargin=10pt,
  innerbottommargin=10pt
]{theorem}

\surroundwithmdframed[
  linewidth=1pt,
  linecolor=gray,
  backgroundcolor=gray!5
]{remark}%\geometry{showframe} % display margins for debugging page layout
\surroundwithmdframed[
  linewidth=1pt,
  linecolor=black,
  backgroundcolor=blue!5
]{lemma}%\geometry{showframe} % display margins for debugging page layout

\usepackage{graphicx} % allow embedded images
  \setkeys{Gin}{width=\linewidth,totalheight=\textheight,keepaspectratio}
  \graphicspath{{graphics/}} % set of paths to search for images
\usepackage{amsmath}  % extended mathematics
\usepackage{booktabs} % book-quality tables
\usepackage{units}    % non-stacked fractions and better unit spacing
\usepackage{multicol} % multiple column layout facilities
\usepackage{lipsum}   % filler text
\usepackage{fancyvrb} % extended verbatim environments
  \fvset{fontsize=\normalsize}% default font size for fancy-verbatim environments

\usepackage{hyperref}
% Standardize command font styles and environments
\newcommand{\doccmd}[1]{\texttt{\textbackslash#1}}% command name -- adds backslash automatically
\newcommand{\docopt}[1]{\ensuremath{\langle}\textrm{\textit{#1}}\ensuremath{\rangle}}% optional command argument
\newcommand{\docarg}[1]{\textrm{\textit{#1}}}% (required) command argument
\newcommand{\docenv}[1]{\textsf{#1}}% environment name
\newcommand{\docpkg}[1]{\texttt{#1}}% package name
\newcommand{\doccls}[1]{\texttt{#1}}% document class name
\newcommand{\docclsopt}[1]{\texttt{#1}}% document class option name
\newenvironment{docspec}{\begin{quote}\noindent}{\end{quote}}% command specification environment

\usepackage{helvet}
\renewcommand{\familydefault}{\sfdefault}
\setcounter{secnumdepth}{2}

%%title
% Make section and subsection headings bold (Tufte-compatible)
\titleformat{\section}%
  [frame]% shape
  {\normalfont\Large\bfseries}% format applied to label+text
  {\thesection}% label
  {1em}% horizontal separation between label and title body
  {}% before the title body
  []% after the title body

\titleformat{\subsection}%
  [hang]% shape
  {\normalfont\large\bfseries}% format applied to label+text
  {\thesubsection}% label
  {1em}% horizontal separation between label and title body
  {}% before the title body
  []% after the title body

\titleformat{\subsubsection}%
  [hang]% shape
  {\normalfont\normalsize\bfseries}% format applied to label+text
  {\thesubsubsection}% label
  {1em}% horizontal separation between label and title body
  {}% before the title body
  []% after the title body

\title{Linear Algebra}
\author[Osman Kagan Yayla]{Osman Kagan Yayla}
\begin{document}

\maketitle% this prints the handout title, author, and date

\begin{abstract}
	This is a document to collect notes on linear algebra.
\end{abstract}
\section{Row Echelon Form}
\begin{definition}[Row Echelon Form]
A matrix $M$ is in \textbf{row echelon form} (REF) if:
\begin{enumerate}

    \item All non-zero rows are above any rows that are all zero.

    \item The leading coefficient (the first non-zero entry from the left, also
	    called the \textbf{pivot}) of each non-zero row is strictly to the
	    right of the leading coefficient of the row above it.

\end{enumerate}
\end{definition}

\begin{example}
\begin{equation*}
\begin{bmatrix}
1 & * & * & * & * & * & * \\
0 & 0 & 2 & * & * & * & * \\
0 & 0 & 0 & 3 & * & * & * \\
0 & 0 & 0 & 0 & 0 & 0 & 0
\end{bmatrix}
\end{equation*}
\end{example}

\begin{definition}[Elementary Row Operations]
The following operations on a matrix are called \textbf{elementary row operations}:
\begin{enumerate}
    \item Subtract a multiple of the $i$-th row from the $j$-th row, where $j \neq i$.
    \item Reorder (permute/exchange) any two rows.
\end{enumerate}
\end{definition}

\begin{theorem}
Any matrix $A \in \mathbb{R}^{m \times n}$ can be converted into row echelon form using only elementary row operations.
\end{theorem}

\begin{proof}
We prove by induction on the number of rows of $A$.

We begin with the general matrix $A$:
\begin{equation*}
\begin{bmatrix}
a_{1,1} & a_{1,2} & \cdots & a_{1,k} & \cdots & a_{1,n} \\
a_{2,1} & a_{2,2} & \cdots & a_{2,k} & \cdots & a_{2,n} \\
\vdots & \vdots & \ddots & \vdots & \ddots & \vdots \\
a_{k,1} & a_{k,2} & \cdots & a_{k,k} & \cdots & a_{k,n} \\
\vdots & \vdots & \ddots & \vdots & \ddots & \vdots \\
a_{m,1} & a_{m,2} & \cdots & a_{m,k} & \cdots & a_{m,n}
\end{bmatrix}
\end{equation*}

\textbf{Base Case:} If $a_{1,1}$ equals zero, exchange the first row with a row that has a nonzero term in the first column. If $a_{1,1}$ does not equal zero, use Gaussian elimination to reduce the rest of the first column to zero. This is done by multiplying row 1 by the ratio of $a_{\ell,1}$ to $a_{1,1}$ and subtracting it from row $\ell$, for all rows $\ell > 1$. The following matrix results:
\begin{equation*}
\begin{bmatrix}
a_{1,1} & a_{1,2} & \cdots & a_{1,k} & \cdots & a_{1,n} \\
0 & a_{2,2} & \cdots & a_{2,k} & \cdots & a_{2,n} \\
\vdots & \vdots & \ddots & \vdots & \ddots & \vdots \\
0 & a_{k,2} & \cdots & a_{k,k} & \cdots & a_{k,n} \\
\vdots & \vdots & \ddots & \vdots & \ddots & \vdots \\
0 & a_{m,2} & \cdots & a_{m,k} & \cdots & a_{m,n}
\end{bmatrix}
\end{equation*}

\textbf{Induction Step:} Assume that the first $k-1$ columns are in row echelon form. We need to show the first $k$ columns can be reduced to row echelon form. We start with the following matrix:
\begin{equation*}
\begin{bmatrix}
a_{1,1} & a_{1,2} & \cdots & a_{1,k-1} & a_{1,k} & \cdots & a_{1,n} \\
0 & a_{2,2} & \cdots & a_{2,k-1} & a_{2,k} & \cdots & a_{2,n} \\
\vdots & \vdots & \ddots & \vdots & \vdots & \ddots & \vdots \\
0 & 0 & \cdots & a_{k-1,k-1} & a_{k-1,k} & \cdots & a_{k-1,n} \\
0 & 0 & \cdots & 0 & a_{k,k} & \cdots & a_{k,n} \\
0 & 0 & \cdots & 0 & a_{k+1,k} & \cdots & a_{k+1,n} \\
\vdots & \vdots & \ddots & \vdots & \vdots & \ddots & \vdots \\
0 & 0 & \cdots & 0 & a_{m,k} & \cdots & a_{m,n}
\end{bmatrix}
\end{equation*}

If $a_{k,k}$ equals zero, exchange the row with a row with a nonzero term in the $k$-th column. If $a_{k,k}$ does not equal zero, use Gaussian elimination to reduce the rest of the $k$-th column to zero. This is done by multiplying row $k$ by the ratio of $a_{\ell,k}$ to $a_{k,k}$ and subtracting it from row $\ell$, for all rows $\ell > k$. The following matrix results:
\begin{equation*}
\begin{bmatrix}
a_{1,1} & a_{1,2} & \cdots & a_{1,k-1} & a_{1,k} & \cdots & a_{1,n} \\
0 & a_{2,2} & \cdots & a_{2,k-1} & a_{2,k} & \cdots & a_{2,n} \\
\vdots & \vdots & \ddots & \vdots & \vdots & \ddots & \vdots \\
0 & 0 & \cdots & a_{k-1,k-1} & a_{k-1,k} & \cdots & a_{k-1,n} \\
0 & 0 & \cdots & 0 & a_{k,k} & \cdots & a_{k,n} \\
0 & 0 & \cdots & 0 & 0 & \cdots & a_{k+1,n} \\
\vdots & \vdots & \ddots & \vdots & \vdots & \ddots & \vdots \\
0 & 0 & \cdots & 0 & 0 & \cdots & a_{m,n}
\end{bmatrix}
\end{equation*}

The first $k$ columns are in row echelon form and by induction, the entirety of matrix $A$ can be reduced to row echelon form.
\end{proof}


\begin{definition}[Pivot]
If $A \in \mathbb{R}^{m \times n}$ is in REF, then the leading non-zero coefficient in each row is called a \textbf{pivot}.
\end{definition}

\begin{lemma}
The number of pivots is at most $\min\{m, n\}$.
\end{lemma}

\begin{proof}
By picture in class.
\end{proof}

\section{System of Linear Equations}

Input $Ax=b$, and the output we are looking for is


We are given $Ax = b$. Let us first consider $b = 0$ (the homogeneous case). Doing elementary row transformations, we obtain $Ux = 0$. Observe $c = 0$.

\begin{example}
Consider the matrix:
\begin{equation*}
A = \begin{bmatrix}
1 & 3 & 3 & 2 \\
2 & 6 & 9 & 5 \\
-1 & -3 & 3 & 0
\end{bmatrix}
\end{equation*}

Applying row operations:
\begin{equation*}
\begin{bmatrix}
1 & 3 & 3 & 2 \\
2 & 6 & 9 & 5 \\
-1 & -3 & 3 & 0
\end{bmatrix}
\rightarrow
\begin{bmatrix}
1 & 3 & 3 & 2 \\
0 & 0 & 3 & 1 \\
0 & 0 & 6 & 2
\end{bmatrix}
\rightarrow
\begin{bmatrix}
1 & 3 & 3 & 2 \\
0 & 0 & 3 & 1 \\
0 & 0 & 0 & 0
\end{bmatrix}
\end{equation*}

Pivots are 1 and 3. Solutions to $Ax = 0$ are the same as the solutions to $Ux = 0$.
\begin{equation*}
\begin{bmatrix}
1 & 3 & 3 & 2 \\
0 & 0 & 3 & 1 \\
0 & 0 & 0 & 0
\end{bmatrix}
\begin{bmatrix}
x_1 \\ x_2 \\ x_3 \\ x_4
\end{bmatrix}
= 0
\end{equation*}

\textbf{Basic variables} are $x_1$ and $x_3$ (with pivots). \textbf{Free variables} are $x_2$ and $x_4$. Pick any value for free variables and solve for the basic variables.

From the second row: $3x_3 + x_4 = 0 \Rightarrow x_3 = -\frac{x_4}{3}$

From the first row: $x_1 + 3x_2 + 3x_3 + 2x_4 = 0 \Rightarrow x_1 = -3x_2 - x_4$

Therefore:
\begin{equation*}
x = \begin{bmatrix} x_1 \\ x_2 \\ x_3 \\ x_4 \end{bmatrix}
= \begin{bmatrix} -3x_2 - x_4 \\ x_2 \\ -\frac{x_4}{3} \\ x_4 \end{bmatrix}
= x_2 \begin{bmatrix} -3 \\ 1 \\ 0 \\ 0 \end{bmatrix}
+ x_4 \begin{bmatrix} -1 \\ 0 \\ -\frac{1}{3} \\ 1 \end{bmatrix}
\end{equation*}

Any $x$ of this form must be a solution of $Ax = 0$ and any solution to $Ax = 0$ must be of this form.
\end{example}
\end{document}
